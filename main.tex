\documentclass{article}
\usepackage[utf8]{inputenc}
\begin{document}

\section{Abstract}
This is just the abstract, which explains what it is that this project
is about.

%%% Local Variables:
%%% mode: latex
%%% TeX-master: "main"
%%% End:



\section{Literature Review}
\subsection{Reinforcement Learning}
\subsubsection{Definitions}
Agent
Environment
State
Action
Reward

\subsubsection{Markov Decision Processes}

\subsubsection{The Bellman Equation}
Equation that represents the value of being in a particular state,
based on the reward given for being in that state and the value of the
best possible adjacent state that can be reached by taking one action,
usually multiplied by a discount factor. This results in a recursive
equation, as the value of the state depends on the value of its
surrounding states. Typically, one can start from the goal state and
propagate the state values ``backwards''. .

\subsubsection{Q Learning}
Replace the state value function in the Bellman equation with the Q
function. This can be interpreted as ``action quality''.

\subsubsection{Q learning with neural networks}
Instead of using a table to store the policy, the policy is ``stored''
in a neural network. This has consequences for stability.

\subsubsection{Types of reinforcment learning}
\label{sec:learning-types}



%%% Local Variables:
%%% mode: latex
%%% TeX-master: "../../main"
%%% End:


\section{Problem Formulation}

In a navigation problem, one generally has a destination that you wish
to get your system to. If the environement is deterministic, using
some kind of path planning method like A* works well, as long as the
``obstacles'' are known. When the environement is dynamic and unknown,
this is not as easy. The ``global'' path may still be there, but there
might be many small local obstacles that need to be avoided. A local
avoidance system is therefore needed.

Potential fields might lead to oscillatory behaviour.

Dynamic window assume no sideways velocity, which is difficult in the
context of ocean currents. Also computationally heavy, but works well
with COLREGS.

\section{Questions}
\subsection{What is a Serret Frenet Frame?}
Signes master

\subsection{What is set-based control?}
Signes PhD says:


Signes
\subsection{how can you show that set-based control laws will always be followed?}




The project has the following steps:
\begin{enumerate}
\item Perform a literature review on the topic of marine surface
  vessel modeling and control, as well as deep learning techniques
  used in tandem with control problems.
\item Set up a simulation environment for the surface vessel,
  incorporating difference levels of disturbances.
\item Set up a baseline using the established control algorithms
\item Apply reinforcement learning to the system and figure out how to
  train the system effectively.
\item Analyse the performance of the machine learning controllers,
  compare the computation time (the expensiveness) between the two
  algorithms, and explore the stability consequences of using machine
  learning techniques in control.  machine learning
\end{enumerate}



\end{document}

%%% Local Variables:
%%% mode: latex
%%% TeX-master: t
%%% End:
