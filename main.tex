\documentclass[titlepage]{elsarticle}
\usepackage[utf8]{inputenc}
\usepackage{natbib}

\title{Machine Learning Controllers for Autonomous Surface Vessels}
\author{Haakon Robinson}
\date{November 2018}

\begin{document}
\maketitle

% \pagenumbering{roman}

\frontmatter
\begin{abstract}
  %/This is some real/% shit
\end{abstract}


\pagebreak
\tableofcontents
\pagebreak

%\pagenumbering{arabic}
%\setcounter{page}{1}

\mainmatter
\section{Introduction}
\label{sec:intro}
Study the possibilities of applying reinforcement learning to controls
problems involving marine surface vessels.



\section{Background}
\label{sec:bg}

\subsection{Modeling Marine Surface Vessels}
\label{sec:bg-marine}
% The 6 DOF model
% The 3 DOF model
% Some control laws

\subsection{Deep Learning and Decision Making}
\label{sec:bg-ml}
% Good old fashioned AI (symbolic artificial intelligence)
% Neural network
% Deep neural network

\subsection{Reinforcement Learning}
\label{sec:bg-rl}
% Markov Decision processes
% Bellman equation
% Q learning
% Deep learning
\subsubsection{Definitions}
Agent
Environment
State
Action
Reward

\subsubsection{Markov Processes}
A Markov Process is a model that describes the possibly states of the
environment, and the probability of transitioning from one state to
another. The state is typically a vector of numbers. The transitions
are stochastic, such that at every time step there is a probability of
transitioning to another state. The transition can also map a state
onto itself.

The state must satisfy the \textbf{Markov Property}, also known as the
memoryless property. This means that the information contained in the
state must be independent of past states. In other words, the state
must ``summarise'' the situation, telling you everything you need to
know.

\textbf{The discrete case?}
\textbf{Continuous case?}



\subsubsection{Markov Decision Processes}
A Markov Decision Process (or MDP) is a mathematical tool that can be
used to model decision making. It extends the notion of Markov Chains
by adding the notions of choice and reward.

\subsubsection{The Bellman Equation}
Equation that represents the value of being in a particular state,
based on the reward given for being in that state and the value of the
best possible adjacent state that can be reached by taking one action,
usually multiplied by a discount factor. This results in a recursive
equation, as the value of the state depends on the value of its
surrounding states. Typically, one can start from the goal state and
propagate the state values ``backwards''. .

\subsubsection{Q Learning}
Replace the state value function in the Bellman equation with the Q
function. This can be interpreted as ``action quality''.

\subsubsection{Q learning with neural networks}
Instead of using a table to store the policy, the policy is ``stored''
in a neural network. This has consequences for stability.




%%% Local Variables:
%%% mode: latex
%%% TeX-master: "../main"
%%% End:




\section{Literature Review}
\label{sec:lit}

\subsection{Deep Reinforcement Learning}
\label{sec:lit-dl}
(Human level control through deep reinforcement learning)

\subsection{Set-Based Tasks for Controls}
\label{sec:lit-set-tasks}
Framework for implementing several modes of operation based on the
current active tasks. Allows for the use of several different control
systems at once, each one specialising in different combinations of
tasks. This is a good way to combine the strengths of rule-based and
ML-based controllers.



\section{Problem Formulation}
\label{sec:formulation}
In a navigation problem, one generally has a destination that you wish
to get your system to. If the environment is deterministic, using
some kind of path planning method like A* works well, as long as the
``obstacles'' are known. When the environment is dynamic and unknown,
this is not as easy. The ``global'' path may still be there, but there
might be many small local obstacles that need to be avoided. A local
avoidance system is therefore needed.

Potential fields might lead to oscillatory behavior.

Dynamic window assume no sideways velocity, which is difficult in the
context of ocean currents. Also computationally heavy, but works well
with COLREGS.



\section{Simulation and Testing Environment}
\label{sec:sim+test}



\section{Deep Learning Models}
\label{sec:learning-models}



\section{Training, Testing, and Validation}

\label{sec:training}


\section{Thoughts}
\textbf{Search keywords:}
Neural computer
Program induction
Reinforcement learning
Reinforcement learning
Deepmind
openai
Vinyals pointer network


The project has the following steps:
\begin{enumerate}
\item Perform a literature review on the topic of marine surface
  vessel modeling and control, as well as deep learning techniques
  used in tandem with control problems.
\item Set up a simulation environment for the surface vessel,
  incorporating difference levels of disturbances.
\item Set up a baseline using the established control algorithms
\item Apply reinforcement learning to the system and figure out how to
  train the system effectively.
\item Analyse the performance of the machine learning controllers,
  compare the computation time (the expensiveness) between the two
  algorithms, and explore the stability consequences of using machine
  learning techniques in control.
\end{enumerate}
%%% Local Variables:
%%% mode: latex
%%% TeX-master: "main"
%%% End:



\end{document}

%%% Local Variables:
%%% mode: latex
%%% TeX-master: t
%%% End:
